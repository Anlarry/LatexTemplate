\documentclass[UTF8]{ctexart}
\usepackage[utf8]{inputenc}
\usepackage[T1]{fontenc}


\usepackage{ctex}
\usepackage{minted}
\usepackage{graphicx}
\usepackage{pdflscape}
\usepackage{titlesec}
\usepackage{float}
\usepackage{color}
\usepackage{amsmath}
\usepackage[export]{adjustbox}
\usepackage{fancyhdr}
\usepackage{natbib}

   

\usepackage[colorlinks, 
            linkcolor=black,
            anchorcolor=black,
            citecolor=black
]{hyperref}
% \definecolor{halfgray}{gray}{0.55}

\setcounter{secnumdepth}{4}

\usepackage{geometry}
\geometry{a4paper, left=2.5cm, right=2.5cm, top=2.0cm, bottom=2.0cm}
\pagestyle{fancy}
\lhead{}
\rhead{\small \leftmark \ \textbf{\textcolor[rgb]{0.2,0.2,0.2}{|}} \thepage}
\renewcommand{\headrulewidth}{0pt}



% \titleformat{\paragraph}
% {\normalfont\normalsize\bfseries}{\theparagraph}{1em}{}
% \titlespacing*{\paragraph}
% {0pt}{3.25ex plus 1ex minus .2ex}{1.5ex plus .2ex}


\begin{document}

\bibliographystyle{plain}

\renewcommand{\contentsname}{Contents}
\renewcommand{\bibname}{reference}

\renewcommand{\today}{\number\year-\number\month-\number\day}

\title{{\Huge 生物大数据\linebreak\linebreak}}
%please write your name, Student #, and Class # in Authors, student ID, and class # respectively
\author{\Large 2018302278-王嘉利 \\ 陕西,西安}
\date{\today}
\maketitle
\setcounter{secnumdepth}{4}


\section*{摘要} 

机器学习在迅速发展,出现了非常多优秀的算法和思想。在应用中往往会了解核心思想,但却忽略了其算法背后的数学原理和细节。
在本文中对机器学习的经典算法进行深入分析和讨论,主要包括支持向量机、回归模型以及谱聚类。从数学的角度深入理解这些算法,
并利用测试数据将算法可视化,与背后的数学细节相联系。



\section{支持向量机(support vector machines,SVM)}

支持向量机是一种二分类模型,它的基本模型是定义在特征空间上的间隔最大的线性分类器,间隔最大使它有别于感知机;
SVM还包括核技巧,这使它成为实质上的非线性分类器。SVM的的学习策略就是间隔最大化,可形式化为一个求解凸二次规划的问题,
也等价于正则化的合页损失函数的最小化问题。SVM的的学习算法就是求解凸二次规划的最优化算法。



\bibliography{ref}

\end{document}
