\documentclass[UTF8]{ctexart}


\usepackage{ctex}
\usepackage{minted}
\usepackage{graphicx}
\usepackage{pdflscape}
\usepackage{titlesec}
\usepackage{float}
\usepackage{color}
\usepackage{amsmath}
\usepackage[export]{adjustbox}
\usepackage{fancyhdr}
\usepackage{natbib}

   

\usepackage[colorlinks, 
            linkcolor=black,
            anchorcolor=black,
            citecolor=black
]{hyperref}
% \definecolor{halfgray}{gray}{0.55}

\setcounter{secnumdepth}{4}

\usepackage{geometry}
\geometry{a4paper, left=2.5cm, right=2.5cm, top=2.0cm, bottom=2.0cm}
\pagestyle{fancy}
\lhead{}
\rhead{\small \leftmark \ \textbf{\textcolor[rgb]{0.2,0.2,0.2}{|}} \thepage}
\renewcommand{\headrulewidth}{0pt}



% \titleformat{\paragraph}
% {\normalfont\normalsize\bfseries}{\theparagraph}{1em}{}
% \titlespacing*{\paragraph}
% {0pt}{3.25ex plus 1ex minus .2ex}{1.5ex plus .2ex}


\begin{document}

\bibliographystyle{plain}

\renewcommand{\contentsname}{Contents}
\renewcommand{\bibname}{reference}

\renewcommand{\today}{\number\year-\number\month-\number\day}

\title{{\Huge 人工智能与虚拟助理、聊天机器人\linebreak\linebreak}}
%please write your name, Student #, and Class # in Authors, student ID, and class # respectively
\author{\Large 王嘉利 \\}
\date{\today}
\maketitle
\setcounter{secnumdepth}{4}
\tableofcontents


\section*{摘要} 

在人工智能快速发展的时代,机器视觉、自然语言处理等是当今时代的热点话题,
虚拟助理和聊天机器人是自然语言处理的中重要的应用分支,会话式的AI越来越受到
人们的关注,一方面聊天机器人是图灵测试的一种实现方式,另一方面,
各大互联网巨头研发各自的聊天机器人。虽然,在如今的技术和市场条件下,虚拟助理和
聊天机器人的实际应用远远低于人们的预期,但是我们有理由相信发展AI虚拟助理、聊天机器人是
仍会是未来的热点话题,是发展的大势所趋\cite{谢剑超2018发展}。


\section{应用现状}

AI虚拟助理、聊天机器人的应用市场非常广泛,并且是多元化的,
主要应用娱乐、智能家居、客户服务、教育等领域。聊天机器人的研究源于图灵测设,
而最早的聊天机器人诞生于1966年,由麻省理工学院的Joseph Weizenbaum开发,用于模拟心理医生,并且不断发展。


现如今,我们每个人都可以
感受聊天机器人,各个互联网巨头都推出了各自的人工智能
虚拟助理,比如苹果Siri、谷歌语音搜索Google Now、亚马逊ALEXA、微软CORTANA、
FacebookM、小米小爱同学等等。通过语音和聊天机器人系统进行交互,完成个人事物的查询,
比如天气查询、收发短信、日程提醒、智能搜索、播放音乐,给人机交互带来新的体验。

在教育领域,聊天机器人可以帮助用户学习语言,通过聊天机器人增强学生和教师之间的互动和参与感,
在科罗拉多州立大学,使用虚拟学习助手,给学生短文打分,提供互动的一对一教学辅导;
在医疗行业,聊天机器人可以降低护理人员的工作负担;在银行等需要客户服务的行业,虚拟助理和聊天机器人可以
提升用户体验、降低人力成本;在娱乐场景,可以用于提升游戏虚拟效果。

\section{核心问题和主要技术}

一般来讲,聊天机器人系统包括五个模块:
\begin{itemize}
    \item \textbf{语音识别} \ 将语音转化为文字交给自然语言理解模块处理
    \item \textbf{自然语言理解} \ 理解用户输入语义,输出到对话管理模块
    \item \textbf{对话管理} \ 协调各个模块,维护当前对话状态;输出到自然语言生成模块
    \item \textbf{自然语言生成} \ 生成回复文本,输出到语音合成模块
    \item \textbf{语音合成} \ 输出音频信号
\end{itemize}

这里主要针对自然语言理解和对话管理进行讨论。自然语言理解模块要解决的问题是理解
文本语义,通常,将文本处理转化为词向量或句子向量,将非结构化数据转化为可计算的结构化数据,通过构建基于上下文的词向量,
从而做到提取文本信息,再交给对话管理模块\cite{杨晔2020基于深度学习的聊天机器人的研究}。

\subsection{word2vec}

word2vec 是词嵌入(word embedding)的方式之一,由Google在2013提出的一套新的词嵌入方法。
通过CBOW(Continuous Bag-of-Words Model)和Skip-gram(Continuous Skip-gram Model)两种方法训练。

\begin{itemize}
    \item \textbf{CBOW} 输入某个特征词的上下文相关词对应的词向量,输入则是这个特征词的词向量
    \item \textbf{Skip-gram} 与CBOW相反,输入特定词的词向量,输出特定词上下文词向量
\end{itemize}

word2vec会考虑到词的上下文,词嵌入效果好,而且训练速度快、通用性强。但是另一方面,
由于词和向量是一对一的关系,难以解决同义词带来的问题。\cite{吴威震2019基于}

\subsection{seq2seq+Attention} 

获得自然语言的向量化表示后,还需要生成回复,经典的框架是Encoder-Decoder,seq2seq with Attention是其中一个应用
广泛的模型,思路简单、容易扩展。seq2seq模型是从序列到序列的过程,
对词向量编码和分解,获得对话中的特征向量,并利用Attention模型根据相似度得到相应的答案。

seq2seq模型主要由2部分组成,包括编码器(对应问句)和解码器(对应回复语句)。而Attention模型是模拟人脑注意力的模型,用来处理序列相关数据。
\cite{简治平2019基于}


\section{困境与挑战}

虽然目前已经可以看到一些虚拟助理和聊天机器人的应用,但是其研究仍然面临很大的挑战,比如对话
上下文、对话策略、虚拟助理的智能程度、用户隐私等。

\subsection{智能程度}

虽然到处都有各种虚拟助理、聊天机器人产品,但是却并被广泛引用,一个很重要的因素就是虚拟助理的
智能程度还不够。首先,人与人之间的对话,话题是可以轻易改变的,这就对聊天机器人理解上下文的能力由很高的
要求;其次,在对话中聊天机器人难以理解所有的自然语言,而且难以人的习惯回复。

这就导致现在流行的虚拟助理和聊天机器人都显得有些鸡肋,只要和它几次对话就能发现对方是机器人,不能通过图灵测试。
因此研究虚拟助理、聊天机器人不仅在商业有重要意义,在学术界也是对图灵测试的一个挑战。
Facebook虚拟助手M在2015年推出,但是却在2018年结束,很大一个原因就是智能程度不够高,很多应该由虚拟助手完成的任务
,都需要人力解决,任务自动化程度低。

\subsection{上下文和情绪}

自然语言是非常灵活的,同一句话在不同的上下文中可能会出现截然不同的意思,而且一句话的含义会受到人所处环境的影响,
也会受到对话双方关系的影响。说话的语气、表情、情绪也都是对话中的重要信息,然而目前流行的虚拟助理和聊天机器人
和真正的人与人交互还有很远的距离。

\subsection{用户体验以及隐私}

亚马逊的Echo出现过侵犯用户隐私的问题。虚拟助理
、聊天机器人是直接和人进行交互的,如何提升用户体验并且保护用户隐私是必须被考虑的问题。
如果做不到这两点,一个虚拟助手产品将难以广泛应用。

比如,Amazon的聊天机器人Echo曾在一个用户睡觉时发出很大的声音并且发出毛骨悚然的笑声,这会对虚拟助理、聊天机器人的发展带来阻力。

\section{机遇与未来}

虚拟助理和聊天机器人虽然目前面临极大的困境,但也存在这机遇和挑战。
在人工智能时代,微软提出Conversation As A Platform的理念,因为人类最自然的交互方式
是通过自然语言,而现在却不得不通过鼠标和按键与机器交互,虚拟助理将带来新的人机交互方式。
可以想象,虚拟助理、聊天机器人会渗透到各行各业。

在教育、医疗、服务、娱乐、智能家居等行业都可以看到虚拟助理的应用。甚至是在出版业,出版商通过智能的聊天机器人,可以提升用户的购书体验,更好地刻画用户画像,
推荐图书充分考虑用户的
个人喜好\cite{杨扬2020智能聊天机器人技术在出版业的创新应用及发展趋势}。另外也可以与作者交互,借助聊天机器人探索作者和读者的交互。
随着技术不断发展,聊天机器人也会迈向虚拟生命,核心在于模拟生命主要特征、以多形态和多模态进行交互,人机交互的方式会变得更加丰富。
\cite{邱楠2017从}


而为未来的研究将会着重在:端到端的对话系统,用统一的模型代替序列化执行自然语言理解、对话管理等步骤;从特定域到开放域;更加关注机器人情商,机器人的个性化性感\cite{张伟男2016}。

\bibliography{ref1}

\end{document}
